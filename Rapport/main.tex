\documentclass[a4paper, 10pt, twocolomn]{article} %ou report.... ça change surtout la numérotation des sections !
\usepackage{import}
\usepackage{preambule}
\usepackage{coding}

%\usepackage{makeidx}
%\makeindex

\title{Challenge\\
	Otto}
\author{Chaabane Imad, Daoudi Nassim, Estrade Victor\\
   M2 AIC Paris-Saclay\\
   \texttt{imad-eddine.chaabane@u-psud.fr}\\
   \texttt{nassim.daoudi@u-psud.fr}\\
   \texttt{victor.estrade@u-psud.fr}\\
        }
\date{\today}
 

\begin{document}
\maketitle
\tableofcontents
\clearpage


\section*{Introduction} % (fold)
\label{sec:introduction}

\todo Introduire ... huummm ! 

% section introduction (end)

\section{Les données} % (fold)
\label{sec:les_donn_es}

\subsection{Distribution} % (fold)
\label{sub:distribution}

\begin{figure}
 \centering
 \includegraphics[scale=0.6]{Files/Class_distrib/Class_distribution_of_feature_53.png}
 \includegraphics[scale=0.6]{Files/Class_distrib/Class_distribution_of_feature_13.png}
 \caption[Distribution des attributs]{Distribution des attributs}
 \label{fig: distrib}
\end{figure}

Tous les attributs ont globalement la même distributions, beaucoup de 0 et 
en dehors des zéros une distribution ressemblant aux figures de 
\ref{fig: distrib} (L'intégralité des distributions sont disponnibles dans l'archive).

% subsection distribution (end)

\subsection{Correlations} % (fold)
\label{sub:correlations}

\todo On cherche s'il y a des correlations fortes entre les différents
attributs

\begin{figure}
 \centering
 \includegraphics[scale=0.6]{Files/covar.png}
 \caption[Possible dependencies]{Possible dependencies}
 \label{fig: covar}
\end{figure}

\todo correlation vs dependency

\url{http://www.stat.cmu.edu/~cshalizi/uADA/13/reminders/uncorrelated-vs-independent.pdf}

% subsection correlations (end)
\clearpage

\subsection{L'évaluation} % (fold)
\label{sub:l_valuation}

L'une des particularités du challenge Otto est la métrique d'évaluation,
"multi-class logarithmic loss", qui récompense les réponses "approximative"
donnant l'avantage à un modèle exprimant peu de confiance dans les instances 
qu'il a du mal à classifier. 

Là où l'erreur ne tient pas compte de la confiance du modèle.


% subsection l_valuation (end)
% section les_donn_es (end)

\section{Dry runs} % (fold)
\label{sec:dry_runs}

Afin d'avoir une idée de la difficulté du problème nous avons commencé par
lancer quelques modèles simples sur les données.

\todo Des arbres de décision car ils sont rapides et simples à entraîner et ils
sélectionnent bien les attributs.


\todo Le problème consistant à optimiser des probabilités nous nous sommes
naturellement tourné vers Naive Bayes. 



% section dry_runs (end)

\section{Nos modèles} % (fold)
\label{sec:nos_mod_les}

\todo Quels modèles ? Pourquoi eux ?

% section nos_mod_les (end)

\section{Comparaison} % (fold)
\label{sec:comparaison}

\todo Super plot !

\todo Qui est mieux ? Pourquoi ?

On remarque que nos modèles ont une variance bien plus importante que 
les modèles de sklearn. Les mesures de performances sont donc très incertaine.

% section comparaison (end)

\section*{Conclusions} % (fold)
\label{sec:conclusions}

\todo Pourquoi on déchire

% section conclusions (end)

\end{document}
